% !TEX root = ../../math6370.tex

\section{Units in $\O_K$}
\subsection{Roots of Unity}

Fix a number field $K$ of degree $n$ with $r$ real embeddings and $s$ pairs of complex embeddings, i.e. $2s$ total complex embeddings. We now work on addressing the following question: what are the units of the number ring $\O_K$?

Consider the map
	\[
	\begin{tikzcd}
	\phi: K^ \times \arrow{r} & \I_K \\[-3ex]
	\alpha \arrow[maps to]{r} & \alpha \O_K
	\end{tikzcd}
	\]
where $\I_K$ is the group of fractional ideals of $\O_K$ which, by the factorization of ideals in $\O_K$ into prime ideals, is the free abelian group on the prime ideals of $\O_K$ The cokernel of $\phi$ is the class group:
	\[
	\coker \phi = \qfrac{\I_K}{\im \phi} = \qfrac{\I_K}{\P_L}= \Cl_K
	\]
But what is the kernel of $\phi$? If $\alpha \O_K= \O_K$ as fractional ideals, then $1 \in \alpha \O_K$. But then there is some $\beta \in \O_K$ such that $\alpha \beta =1$. Therefore, $\ker \phi= \O_K^\times$ is the group of units of $\O_K$. 


\begin{dfn}[Roots of Unity]
The group $\mu_K$ of roots of unity of $\O_K$ is the torsion subgroup of $\O_K^\times$, i.e. the group of $\alpha \in \O_K^\times$ such that $\alpha^m=1$ for some integer $m>0$. 
\end{dfn}

Define the homomorphism $\phi: \O_K^\times \to \R^{r+s}$ by
	\[
	\alpha \mapsto \big( \log |\sigma_1(\alpha)|, \ldots, \log |\sigma_r(\alpha)|, 2 \log |\sigma_{r+1}(\alpha)|, \ldots, 2 \log |\sigma_{r+s}(\alpha)| \, \big)
	\]

\begin{prop}
The group $\psi(\O_K^\times)$ is discrete in $\R^{r+s}$. The kernel of $\psi$ is the finite group $\mu_K$. 
\end{prop}

\pf Take any $C \geq 1$. Consider the set $S$ of $\alpha \in \O_K^\times$ such that $-C \leq \log |\sigma_i(\alpha)| \leq C$ for all $1 \leq i \leq r+s$. We claim that $S$ is a finite set. If $\alpha \in S$, then $|\sigma_i(\alpha)| \leq e^C$ for all $i$. Since $i: \O_K \to \R^n$ has a discrete image, there are only finitely many $\alpha \in \O_K$ with $|\sigma(\alpha)| \leq e^C$ for all $\sigma: K \hra \C$. Hence, $S$ is finite.

Since $C$ is arbitrary and $S$ is finite, $\psi(\O_K^\times)$ is discrete. Moreover, $\ker \psi \subseteq S$ for any $C$. Since $S$ is finite, $\ker \psi$ is a finite subgroup of $\O_K^\times$. Hence, $|ker \psi \subseteq \mu_K$. Now let $\zeta \in \mu_K$ be such that $\zeta^m=1$. Then for any embedding $\sigma: K \hra \C$, $|\sigma(\zeta)^m|=1$, so $|\sigma(\zeta)|=1$ and $\zeta \in \ker \psi$. Hence, $\mu_K \subseteq \ker \psi$. But then $\mu_K= \ker \psi$. \qed \\




\subsection{Dirichlet's Unit Theorem \label{sec:unit}}


We hope to prove Dirichlet's Unit Theorem:

\unit

That is, there are $\mu_1,\ldots,\mu_{r+s-1} \in \O_K^\times$ such that every $\alpha \in \O_K^\times$ is of the form $\alpha= \zeta \cdot \mu_1^{n_1} \cdots \mu_{r+s-1}^{n_{r+s-1}}$.

Fix a number field $K$ of degree $n$ with $r$ real embeddings and $2s$ complex embeddings, i.e. $s$ pairs of complex conjugate pairs of embeddings. We wish to study $\O_K^\times$. Recall that $\I_K$ is the group of fractional ideals of $\O_K$; that is, $\I$ is the free abelian group generated by the prime ideals of $\O_K$. Consider the map $\phi: K^\times \to \I_K$ given by $\alpha \mapsto \alpha \O_K$. The cokernel of $\phi$ is precisely the class group:
	\[
	\coker \phi = \qfrac{\I_K}{\im \phi} = \qfrac{\I}{\P(K)}= \Cl_K. 
	\]
That is, $\phi$ is `not far' from being surjective. What is the kernel of $\phi$? We have $\ker \phi= \O_K^\times$: if $\alpha \O_K=\O_K$ as fractional ideals, then $\alpha\beta=1$ for some $\beta \in \O_K$. 

\begin{dfn}[Roots of Unity]
Let $\mu_K$ be the torsion subgroup of $\O_K^\times$; that is, $\mu_K$ is the (finite) group of $\alpha \in \O_K^\times$ such that $\alpha^n=1$ for some $n>0$ (along with the identity). We call $\mu_K$ the roots of unity of $\O_K$. 
\end{dfn}

Define a homomorphism $\psi: \O_K^\times \to \R^{r+s}$ by
	\[
	\alpha \mapsto \big( \log|\sigma_1(\alpha)|, \ldots, \log|\sigma_r(\alpha)|, 2\log|\sigma_{r+1}(\alpha)|, \ldots, 2\log|\sigma_{r+s}(\alpha)| \big).
	\]
This is a homomorphism as $\sigma$ and $|\cdot|$ respect multiplication and the logarithm turns multiplication to addition. 

\begin{prop}
$\phi(\O_K^\times)$ is discrete in $\R^{r+s}$. The kernel of $\psi$ is the finite group $\mu_K$. 
\end{prop}

\pf Fix $C \geq 1$. Consider the set $S=\{ \alpha \in \O_K^\times \colon \log|\sigma_i(\alpha)| \leq C \forall 1\leq i \leq r+s\}$. We claim that $S$ is finite: if $\alpha \in S$, then $|\sigma_i(\alpha)| \leq e^C$ for all $1 \leq i \leq r+s$. Since $\iota: \O_K \to \R^n$ has discrete image, there are only finitely many $\alpha \in \O_K$ with $|\sigma(\alpha)| \leq e^C$ for all $\sigma: K \hra \C$. Hence, $S$ is finite. Now $C$ was arbitrary and $S$ was finite, this shows $\psi(\O_K^\times)$ is discrete. 

Now $\ker \psi \subseteq S$ for any $C$. As $S$ is finite, $\ker \psi$ is a finite subgroup of $\O_K^\times$. But then $\ker \psi \subseteq \mu_K$. Let $\zeta \in \mu_J$ such that $\zeta^n=1$. For any embedding $\sigma: K \hra \C$, $|\sigma(\zeta)|^m=1$. But then $|\sigma(\zeta)|=1$ and $\zeta \in \ker \psi$. Therefore, $\mu_K \subseteq \ker \psi$ and hence $\mu_K=\ker \psi$. \qed \\


Before proving Dirichlet's Unit Theorem make a definition, and prove a useful lemma. Define
	\[
	V:= \left\{ x \in \R^{r+s} \;\bigg|\; \sum_{i=1}^{r+s} x_i =0 \right\}
	\]
Note that $V$ is an $\R$-vector space of dimension $r+s-1$. We claim that $\psi(\O_K^\times) \subseteq V$. If $\alpha \in \O_K^\times$, then 
	\[
	1= |\Nm{K/\Q}(\alpha)|= \prod_\sigma |\sigma(\alpha)| = \prod_{i=1}^r |\sigma_i(\alpha)| \cdot \prod_{i=1}^{r+s} |\sigma_i(\alpha)|^2.
	\]
Taking logarithms, we have
	\[
	0= \sum_{i=1}^r \log|\sigma_i(\alpha)| + 2 \sum_{i=1}^{r+s} \log|\sigma_i(\alpha)|.
	\]
But then $\psi(\alpha) \in V$. By the discreteness of $\psi(\O_K^\times)$,
	\[
	\qfrac{\O_K^\times}{\mu_K} \cong \psi(\O_K^\times).
	\]
Dirichlet's Unit Theorem is equivalent to the claim that $\psi(\O_K^\times)$ is a lattice in $V$. This leads to the regulator of a number field.

\begin{dfn}[Regulator]
Given a number field $K$, the regular of $K$ is
	\[
	\reg_K:= \dfrac{\covol(\psi(\O_K^\times))}{\sqrt{r+s}}
	\]
Alternatively, consider the $(r+s-1) \times (r+s)$ matrix $M=(e_j \log|\sigma_j(u_i)|)$, where $u_1,\ldots,u_{r+s-1}$ is a basis for $\O_K^\times/\mu_k$ as an $\O_K^\times$-module. Then $\pm\reg_K$ is the determinant of the matrix $M$ after removing a column, say $M'$. Then $\reg_K= |\det M'|$.
\end{dfn}

\begin{rem}
The regulator is a real positive number. The denominator $\sqrt{r+s}$ is merely a `normalization' factor. 
\end{rem}

It is often useful to know the regulator of a field $K$. Indeed, if $\mu_K, \alpha_1,\ldots,\alpha_{r+s-1}$ generate a subgroup $H \leq \O_K^\times$ of rank $r+s-1$ then
	\[
	\dfrac{\covol(\psi(H))}{\covol(\psi(\O_K^\times))}= [\O_K^\times \colon H].
	\]
The denominator we obtain from the regulator after `normalization'. Therefore, if we know the index and the regulator, we can find $\covol(\psi(H))$ and use this to to find $H$. In any case, we are now in a position to prove Dirichlet's Unit Theorem.


\unit*

\pf Consider the case $r+s-1=0$. In this case, $\psi(\O_K^\times) \subseteq K$ and $\dim_\R V=r+s-1=0$. Therefore, $\psi(\O_K^\times)=0$. 

Now assume that $r+s-1>0$. Assume to the contrary that $\psi(\O_K^\times)$ has rank less than $r+s-1$. Then there is a nonzero $\R$-linear map $f: V \to \R$ with $f(\psi(\O_K^\times))=0$. [Since $\psi(\O_K^\times)$ has rank less than $r+s-1$, there are some defining equations. These are then in the kernel] There are unique $c_i \in \R$ such that for all $v \in V \subseteq \R^{r+s}$, $f(v)= c_1v_1+\cdots+c_{r+s-1} v_{r+s-1}$. Note that we do not require $v_{r+s}$ since $\sum_i v_i=0$. Using $\phi(\O_K^\times) \subseteq V$, we can extend $f$ to an $\R$-linear map $f: \R^{r+s} \to \R$. 


Define 
	\[
	A:= \sqrt{|\disc K|} \left(\dfrac{2}{\pi}\right)^s>0,
	\]
which depends only on the number field $K$. We want to construct a sequence $\alpha_1,\alpha_2,\ldots$ of nonzero elements of $\O_K$ such that $|\Nm{K/\Q}(\alpha_i)| \leq A$, and the values $f(\psi(\alpha_i))$ are distinct. Assuming such a sequence exists, then $\Nm{K/\Q}(\alpha_i \O_K) |\Nm{K/\Q}(\alpha_i)| \leq A$. Since there are only finitely many ideals in $\O_K$ of bounded norm, there are distinct $\alpha_n,\alpha_m$ with $\alpha_n \O_K= \alpha_m \O_K$ and $f(\psi(\alpha_n)) \neq f(\psi(\alpha_m))$ (this is the Pigeonhole Principle). Define $\beta= \alpha_n\alpha_m^{-1} \in K^\times$. Then $\beta \O_K=\O_K$ so that $\beta \in \O_K^\times$. On the other hand,
	\[
	f(\psi(\beta))= f(\psi(\alpha_n)) - f(\psi(\alpha_m)) \neq 0,
	\]
contradicting the choice of $f$ such that $f(\psi(\O_K^\times))=0$ as $\beta \in \O_K^\times$.

We need now only prove the existence of such a sequence. Recall we have an embedding $\iota: \O_K \hra \R^n$ such that $\iota(\O_K)$ is a lattice in $\R^n$ with covolume $\sqrt{|\disc K|}$. Define
	\[
	X:= \left\{ x \in \R^n \colon |x_i| \leq b_i \text{ for } 1\leq i \leq r, x_i^2+x_{i+s}^2 \leq b_i^2 \text{ for }r+1 \leq i \leq r+s \right\}
	\]
with $b_i>0$ fixed real numbers such that $b_1\cdots b_r(b_{r+1} \cdots b_{r+s})^2=A$. If $\alpha \in \O_K$ with $\iota(\alpha) \in X$, then $|\sigma_i(\alpha)| \leq b_i$ for $1 \leq i \leq r+s$. Therefore,
	\[
	|\Nm{K/\Q}(\alpha)|= \left| \prod_{i=1}^{r+s} \sigma_i(\alpha) \cdot \prod_{i=1}^s \overline{\sigma_{r+i}}(\alpha) \right| \leq b_1\cdots b_r(b_{r+1}\cdots b_{r+s})^2=A.
	\]


Note that $X$ is compact, symmetric, convex, and
	\[
	\vol X= \prod_{i=1}^r (2b_i) \cdot \prod_{i=r+1}^{r+s} (\pi b_i^2) \cdot 2^s= 2^{r+s} \pi^s A= 2^n \sqrt{|\disc K|} = 2^n \covol(\iota(\O_K)).
	\]
So by Minkowski's Theorem, there is some $\alpha \in \O_K \setminus \{0\}$ such that $\iota(\alpha) \in X$. Therefore, $|\Nm{K/\Q}(\alpha)| \leq A$. By varying the $b_i$, we can find $\alpha$ with $f(\psi(\alpha)) \in \R$ arbitrarily large. This will complete the proof. 


We claim that for $1 \leq i \leq r+s$, $b_i/A \leq |\sigma_i(\alpha)| \leq b_i$. We know that $|\sigma_i(\alpha)| \leq b_i$ by the construction of $\alpha$. For the lower bound, we have
	\[
	\begin{split}
	1 \leq |\Nm{K/\Q}(\alpha)| &= |\sigma_1(\alpha)| \cdots |\sigma_r(\alpha)| \left( |\sigma_{r+1}(\alpha)| \cdots |\sigma_{r+s}(\alpha)| \right)^2 \\
	&\leq |\sigma_i(\alpha)| \; \dfrac{b_1 \cdots b_r(b_{r+1} \cdots b_{r+s})}{b_i} = |\sigma_i(\alpha)| \; \dfrac{A}{b_i},
	\end{split}
	\]
i.e. $|\sigma_i(\alpha)|$ is `roughly' $b_i$. Evaluating $f(v)= \sum_{i=1}^{r+s-1} c_iv_i$ at $\psi(\alpha)$ yields
	\[
	f(\psi(\alpha))= \sum_{i=1}^{r+s-1} c_i e_i \log |\sigma_i(\alpha)|.
	\]
Approximating $|\sigma_i(\alpha)|$ by $b_i$,
	\[
	\begin{split}
	\left| f(\psi(\alpha)) - \sum_{i=1}^{r+s-1} c_ie_i \log b_i \right| &\leq 2 \sum_{i=1}^{r+s-1} |c_i| \, |\log|\sigma_i(\alpha)| - \log b_i | \\
	&\leq 2 \sum_{i=1}^{r+s-1} |c_i| \log(|\sigma_i(\alpha)|/b_i) \\
	&\leq 2\sum_{i=1}^{r+s-1} |c_i| \log A,
	\end{split}
	\]
where the last inequality follows from the remarks above. The bound $2\sum_{i=1}^{r+s-1} |c_i| \log A$ depends only on the number field $K$. Rearranging,
	\[
	f(\psi(\alpha)) \geq \sum_{i=1}^{r+s-1} c_i e_i \log b_i - 2 \sum_{i=1}^{r+s-1} |c_i| \log A.
	\]
The $c_i$ are not all zero since $f \neq 0$. Therefore, there exist $b_1,\ldots,b_{r+s-1}$ to be positive real numbers such that
	\[
	\sum_{i=1}^{r+s-1} c_ie_i \log b_i
	\]
is arbitrarily large. Since $b_{r+s}$ is not present in the sum, we can choose $b_1,\ldots,\hat{b_i},\ldots,b_{r+s-1}$, where $b_i$ is excluded from this list, such that $b_1\ldots b_r(b_{r+1} \cdots b_{r+s})^2=A$. \qed \\





\subsection{Examples of Dirichlet's Unit Theorem}





\begin{ex}
If $K=\Q$, then $r=1$ and $s=0$ so that $r+s-1=0$. Therefore, $\O_\Q^\times=\Z^\times=\{\pm1\}$. \xqed
\end{ex}

\begin{ex}
Let $d>1$ be a squarefree integer. If $K=\Q(\sqrt{d})$, then $r=2$ and $s=0$ so that $r+s-1=1$. Then $\O_K^\times \cong \mu_K \times \Z$. Since $K \subseteq \R$ and the only real roots of unit are $\pm 1$, $\mu_K=\{\pm 1\}$. Under the isomorphism $\O_K \cong \mu_K \times \Z$, there is $\ep \in \O_K^\times$ such that $\O_K^\times=\{ \pm \ep^n \colon n \in \Z\}$. If we choose $\ep>1$, this fundamental unit of this ring of integers. As a few examples:
	\begin{table}[H]
	\centering
	\begin{tabular}{lcr}
	$d$ & $\ep$ & $\Nm{K/\Q}(\ep)$ \\ \hline
	2 & $1+\sqrt{2}$ & $-1$ \\
	10 & $3+\sqrt{10}$ & $-1$ \\
	93 & $\dfrac{29+3\sqrt{93}}{2}$ & $-1$ \\
	94 & $2143295+221064 \sqrt{94}$ & $-1$
	\end{tabular}
	\end{table}
\xqed
\end{ex}


\begin{ex}
Let $d<0$ be a squarefree integer. If $K=\Q(\sqrt{d})$, then $r=0$ and $s=1$ so that $r+s-1=0$. In this case, $\O_K^\times=\mu_K$. We can be even more explicit: if $d \not\equiv 1 \mod 4$, take $\alpha=a+b\sqrt{d} \in \O_K\setminus\{0\}$. Then $\alpha$ is a unit if and only if $\Nm{K/\Q}(\alpha)= \pm 1$ if and only if $a^2-db^2= \pm 1$. This equation only has solutions $(a,b)=(\pm1,0)$ for $d\neq -1$ and  $(a,b)=(0,\pm 1)$ if $d= -1$. Hence, $\alpha= \pm 1$ if $d\neq -1$ and $\alpha= \pm i$ if $d=-1$. Thus, we have found
	\[
	\O_K^\times=
	\begin{cases}
	\{\pm 1\}, & d \neq -1,-3 \\
	\{\pm i\}, & d= -1 \\
	\{\pm 1, \dfrac{\pm 1 \pm \sqrt{-3}}{2} \}, & d= -3
	\end{cases}
	\] \xqed
\end{ex}


\begin{ex}
If $K=\Q(\sqrt[3]{2},\zeta)$, where $\zeta= \frac{-1+\sqrt{-3}}{2}$ is a primitive cube root of unity. We have $[K \colon \Q]=6$. Furthermore, we have $\O_K=\Z[\ep]$. There are no real embeddings, i.e. $r=0$, and six complex embeddings, i.e. $s=3$. Then we have $r+s-1= 2$. Then we have $\mu_K= \mu_6= \{\pm 1, \pm \zeta, \pm \zeta^2\}$. Therefore, $\O_K^\times=\mu_6 \langle \ep,\ov{\ep}\rangle$, where the fundamental unit is
	\[
	\ep= \dfrac{-1 + 2\sqrt[3]{2} + (\sqrt[3]{2})^2}{3} + \dfrac{1 - \sqrt[3]{2} + (\sqrt[3]{2})^2}{2}\; \zeta
	\] \xqed
\end{ex}


\begin{ex}
Let $K=\Q(\sqrt{2},\sqrt{3})$. Then $r=4$ and $s=0$ so taht $r+s-1=3$. The roots of unity of $\O_K$ is $\mu_K=\{ \pm1\}$ with fundamental units
	\[
	\begin{split}
	u_1&= 1+\sqrt{2} \\
	u_2&= \sqrt{2} + \sqrt{3} \\
	u_3&= \dfrac{\sqrt{2}+\sqrt{6}}{2}
	\end{split}
	\]
Verifying that these are units and linearly independent is fairly routine. However, showing that they do indeed generate $\O_K$ is not so trivial. \xqed
\end{ex}


\begin{ex}
Let $K=\Q(\alpha)$, where $\alpha$ is a root of $p_\alpha(x)=x^3-3x+1$. We have $\O_K=\Z[\alpha]$. Now $p_\alpha(x)$ has three real roots so that $r=3$ and $s=0$. Therefore, $\O_K^\times$ has rank $r+s-1=2$. In fact, $\O_K^\times= \pm \langle u_1,u_1 \rangle$, where $u_1= -\alpha+1$ and $u_2=\alpha^2+\alpha-1$.

Given $u= -14+32\alpha+21\alpha^2 \in \O_K^\times$, how do we express $u$ in terms of the generators $u_1,u_2$? Compute the image of $u,u_1,u_2$ under the map $\psi: \O_K^\times \to \R^3$ given by $\alpha \mapsto (\log|\sigma_1(\alpha)|, \log|\sigma_2(\alpha)|, \log|\sigma_3(\alpha)|)$. We have
	\[
	\begin{split}
	\psi(u)&= (-1.0395\ldots, 4.4346\ldots, -3.3950\ldots) \\
	\psi(u_1)&= (-0.4266\ldots, -0.6309\ldots, 1.0575\ldots) \\
	\psi(u_2)&= (-0.6309\ldots, 1.0575\ldots, -0.4266\ldots)
	\end{split}
	\]
Since $u= \pm u_1^mu_2^n$ for unique $m,n \in \Z$ and $\psi$ is a homomorphism, we may write $\psi(u)= m\psi(u_1) + n \psi(u_2)$. A numerical calculation shows $(m,n) \approx (-2.0002\ldots, 3.0001\ldots)$. So we expect $u= \pm u_1^{-2}u_2^3$. A routine calculation verifies that $u= \pm u_1^{-2}u_2^3$. \xqed
\end{ex}



Moreover for any order $R \subseteq K$, we have $R^\times \cong \mu_R \times \Z^{r+s-1}$. Dirichlet proved this for $R=\Z[\alpha]$, i.e. monogenic orders. 

\begin{prop} \label{prop:unitorder}
For any order $R \subseteq K$, $R^\times \cong \mu_R \times \Z^{r+s-1}$, where $\mu_R$ is the set of roots of unity in $R$. 
\end{prop}

\pf Let $N=[\O_K \colon R]$. Note that $N \O_K \subseteq R \subseteq \O_K$. We want to show that $[\O_K^\times \colon R^\times]$ is finite. Consider the map $\phi: \O_K^\times \to (\O_K/N\O_K)^\times$. We want to show $\ker \phi= R^\times$. This would prove finite index. 

Choose $\alpha \in \O_K^\times$. There exists $n \in \N$ such that $\alpha^n \equiv 1 \mod N\O_K$. We have $\alpha^{-1} \equiv 1 \mod N\O_K$. But then both $\alpha-1$ and $\alpha^{-1}-1$ are in $N\O_K \subseteq R$. But then $\alpha,\alpha^{-1} \in R$ so that $\alpha \in R^\times$. Therefore, $[\O_K^\times \colon R^\times]$ divides $\#(\O_K/N\O_K)^\times$. Since $\#(\O_K/N\O_K)^\times$ is finite, this proves finiteness. Furthermore by Theorem~\ref{thm:unit}, $\O_K^\times$ is a finitely generated abelian group of rank $r+s-1$. But then $R^\times$ must too be a finitely generated abelian group of rank $r+s-1$. \qed \\





\subsection{Pell's Equation \label{sec:pell}}





One example of the power of Dirichlet's Unit Theorem deserves special attention: the Pell equation. [One might recall this from Example~\ref{ex:pell2} and Example~\ref{ex:pellgen}.]

Fix $d>0$ a squarefree integer. What are the solutions $(x,y) \in \Z^2$ to the Pell equation $x^2-dy^2=1$? We can rephrase this question as follows: what elements $a+b\sqrt{d} \in K:=\Q(\sqrt{d})$ of the order $R=\Z[\sqrt{d}]$ with $\Nm{K/\Q}(a+b\sqrt{d})=1$? This set is in bijection with the set of solutions of Pell's equation $G:=\{(a,b) \in \Z^2 \colon a^2-db^2=1\}$. We know by Proposition~\ref{prop:unitorder} that $R^\times \cong \mu_R \times \Z$. In this case, $G$ is index 1 (all units have norm 1) or 2 (there is a unit of norm $-1$) in $R^\times$. Therefore, $G= \pm \langle u \rangle \cong \Z/2\Z \times \Z$. 

\begin{ex}
If $d=10$, we have the equation $x^2-10y^2=1$. The fundamental unit of $\O_K=\Z[\sqrt{10}]$ is $\ep=3+\sqrt{10}$. The norm of this fundamental unit is $\Nm{K/\Q}(\ep)= -1$. A few possible solutions are:
	\begin{table}[H]
	\centering
	\begin{tabular}{ccc}
	$n$ & $(\ep^2)^n$ & Solution \\ \hline
	1 & $19+6\sqrt{10}^{\phantom{\int}}$ & (19,6) \\
	2 & $721+228\sqrt{10}$ & (721,228) \\
	3 & $27379+8658\sqrt{10}$ & (27379,8658)
	\end{tabular}
	\end{table}
In fact, these are all the solutions in the positive integers. \xqed
\end{ex}

\begin{ex}
Recall  Example~\ref{ex:pellgen}. We want to find a solution $(x,y)$ in positive integers to $x^2-1141y^2=1$. Let $K=\Q(\sqrt{1141}) \subseteq \R$. The corresponding ring of integers is $\O_K=\Z[\alpha]$, where $\alpha=\frac{1+\sqrt{1141}}{2}$ as $1141 \equiv 1 \mod 4$. Then
	\[
	\O_K= Z[\alpha]= \Z + \Z\alpha= \left\{ \dfrac{a+b\sqrt{1141}}{2} \;\bigg|\; a,b \in \Z, a \equiv b \mod 2 \right\}
	\]
We want to describe $\O_K^\times$ and $R^\times$. Note that $r=2$ and $s=0$ so that $r+s-1=1$. Dirichlet's Unit Theorem, Theorem~\ref{thm:unit}, gives $\O_K^\times= \pm\langle u \rangle$ for some unique fundamental unit $u>1$ in $\O_K^\times$. The problem then reduces down to finding $u$. 

For $\beta \in K^\times$, $\beta\O_K=\O_K$ if and only if $\beta \in \O_K^\times$. We wish to find $\alpha,\beta$ such that $\alpha \O_K=\beta \O_K$, implying $\alpha \beta^{-1} \in \O_K^\times$. As $x^2-x-285 \equiv x^2-x\equiv x(x-1) \mod 3$, factor $3\O_K=\p_3\p_3'$, where $\p_3=(3,\alpha)$ and $p_3'=(3,\alpha-1)$. Similarly, $x^2-x-285 \equiv x(x-1) \mod 5$ so that we factor $5\O_K= \p_5\p_5'$, where $\p_5=(5,\alpha)$ and $\p_5'=(5,\alpha-1)$. Finally, factor
	\[
	\begin{split}
	15-\alpha&= \dfrac{29-\sqrt{1141}}{2} \\
	15+\alpha&= \dfrac{31+\sqrt{1141}}{2} \\
	21-\alpha&= \dfrac{41-\sqrt{1141}}{2}
	\end{split}
	\]
The norms suggest the factorizations as these only involve 3 and 5 --- three ideals, two relations. We have $\Nm{K/\Q}(15-\alpha)= -75= -3 \cdot 5^2$. Note that $15-\alpha \in \p_3$, $15-\alpha \in \p_5$, and $15-\alpha \notin \p_5'$. For the last, note that if $15-\alpha \in \p_5\p_5'=(5)$, then $\frac{15-\alpha}{5} \in \O_K$, a contradiction. Then we have factored $(15-\alpha)=\p_3\p_5^2$. Similarly, we can factor
	\[
	\begin{split}
	(15-\alpha)&= \p_3\p_5^2 \\
	(15+\alpha)&= \p_3^2 \p_5 \\
	(21-\alpha)&= \p_3^3 \p_5
	\end{split}
	\]
Multiplying the last equation by $\p_5^2$ on the right and dividing by $(5)$, we have an equation in fraction ideals
	\[
	\left(\dfrac{21-\alpha}{5}\right) = \p_3^2 \p_5^{-1}.
	\]
To obtain a unit, observe
	\[
	(15-\alpha)^a (15+\alpha)^b \left(\dfrac{21-\alpha}{5}\right)^c= (\p_3^2\p_5^2)^a (\p_3^2\p_5)^b (\p_3^3\p_5^{-1})^c= \p_3^{a+2b+3c} \p_5^{2a+b-c}
	\]
To find a unit in $\O_K$, we need question when this ideal is $\O_K$. This occurs when both exponents vanish, i.e. for $a=\frac{5}{3}c$ and $b= -\frac{7}{3}c$ with $c$ free. Choosing $c=3$, we have $(a,b,c)=(-5,7,-3)$. Then
	\[
	\begin{split}
	\ep&= -(15-\alpha)^{-5} (15+\alpha)^7 \left(\dfrac{21-\alpha}{5}\right)^{-3} \\
	&= 618715978 + 37751109 \,\alpha %constant 1275183065?
	\end{split}
	\]
is a unit in $\O_K$ (which is not $\pm1$). In fact, $\ep$ is the fundamental unit of $\O_K^\times$. Now we wanted solutions of $x^2-1141y^2=1$. Hence, we were searching for units in $R=\Z[\sqrt{1141}] \neq \O_K$. In particular, $\ep \notin R$. However, $\ep^3 \in R^\times$. In fact, $R^\times = \pm \langle \ep^3 \rangle$. This gives a solution $(x_0,y_0)$, where
	\[
	\begin{split}
	x_0&= 1036782394157223963237125215 \\
	y_0&=30693385322765657197397208
	\end{split}
	\]
Moreover, this is the smallest possible positive pair of solutions. \xqed
\end{ex}
