% !TEX root = ../../math6370.tex
\section{Local Fields}
\subsection{$\p$-adic Fields}


\begin{dfn}[Topological Field]
A topological field is a field $K$ with a topology such that $+,-,\cdot: K \to K$ and $(\cdot)^{-1}: K^\times \to K^\times$ are continuous. 
\end{dfn}

\begin{ex}
Both $\R$ and $\C$, under their usual topology, are topological fields. \xqed
\end{ex}

\begin{dfn}[Local Field]
A local field is a topological field with a non-discrete topology that is locally compact, i.e. every point has a neighborhood whose closure is compact. 
\end{dfn}

Now let us see this explicitly in the context of Algebraic Number Theory. Let $K$ be a number field. The fractional ideal generated by $x \in K^\times$ factors uniquely as
	\[
	x \O_K= \prod_{0 \neq \p \subseteq \O_K} \p^{\nu_\p(x)},
	\]
where $\nu_\p(x) \in \Z$. Fix $\p$ and set $\nu_\p(0)=+\infty$. Define $|\cdot|_\p: K \to \R_{\geq 0} \cup \{\infty\}$ via $x \mapsto N(\p)^{-\nu_\p(x)}$.

\begin{dfn}[Absolute Value]
An absolute value on an integral domain $R$ is a function $|\cdot|: R \to \R$ satisfying for all $x,y \in R$,
\begin{itemize}
\item $|x| \geq 0$
\item $|x|=0$ if and only if $x=0$
\item $|xy|= |x|\,|y|$
\item $|x+y| \leq |x|+|y|$
\end{itemize}
\end{dfn}

\begin{ex}
The function constructed above, $|\cdot|_\p: K \to \R_{\geq 0} \cup \{\infty\}$ via $x \mapsto N(\p)^{-\nu_\p(x)}$, is an absolute value on $K$. In fact, we have a stronger triangle inequality:
	\[
	|x+y|_\p \leq \max\{|x|_\p, |y|_\p\}.
	\]
\end{ex}

A usual thing about absolute values is that they give norms which can then be used to give a metric: if $\| \cdot\|$ is a norm then $d(x,y)=\|x-y\|$ is a metric. In particular, $|\cdot|_\p$ induces a topology.

\begin{dfn}[$\p$-adic Topology]
Let $K$ be a number field and $0 \neq \p \subseteq \O_K$. Define the topology induced by the norm $|\cdot|_\p$ is called the $\p$-adic topology. 
\end{dfn}

\begin{rem}
Note that in the trivial number field case $K=\Q$, the topology given by the norm $|\cdot|_\p$ is \emph{not} the usual topology on $\Q$. In particular in this new topology, all triangles are isosceles and the intersection of open balls $B_\ep(x)=\{y \in K \colon |x-y|_\p<\ep \}$ is either empty or is one of the open neighborhoods, i.e. one contains the other. 
\end{rem}

Now that we have a metric topology, we may complete the space so that all Cauchy sequences converge. Let $\widehat{K}$ be the completion of $K$ with respect to the $\p$-adic topology. [Recall the completion of a metric space is the set of equivalence classes of Cauchy sequences in the metric space.] If $\alpha \in \widehat{K}$ is the equivalence class of the Cauchy sequence $(a_n)_{n \in \N}$, then $|\alpha|_\p = \lim_{n \to \infty} |a_n|_\p$. 

\begin{dfn}
Given a number field $K$ and a nonzero prime $\p \subseteq \O_K$, we denote by $K_\p$ the completion of $K$ with respect to $|\cdot|_\p$. The space $K_\p$ is called the $\p$-adic completion of $K$.
\end{dfn}

Much to the hopes and dreams of all Calculus students, here the Divergence Test become the `Convergence Test', i.e.


\begin{prop}
The series $\displaystyle\sum_{n=1}^\infty a_n$ converges if and only if $a_n \to 0$ as $n \to \infty$.
\end{prop}

\noindent \emph{Proof (Sketch): }
	\[
	\left| \sum_{n=m}^N a_n \right| \leq \max\{ |a_n| \colon m \leq n \leq N \}
	\]
\qed \\


\begin{ex}
Consider the 2-adic topology on $\Q$, denoted $\Q_2$. In the field $\Q_2$, the sequence
	\[
	a_n= 1+2 + 2^2 + \cdots + 2^n = \dfrac{2^{n+1} -1}{2-1}= 2^{n+1}-1
	\]
converges to $-1$ as $n \to \infty$. \xqed
\end{ex}


\begin{ex}
For $n \geq 0$, consider the rational number
	\[
	\binom{1/2}{n} = \dfrac{\frac{1}{2} (\frac{1}{2}-1) (\frac{1}{2}-2) \cdots (\frac{1}{2} - n + 1)}{n!}.
	\]
We claim that the denominator is a power of 2: take any odd prime $p$, it suffices to show that
	\[
	\left| \binom{1/2}{n} \right| \leq 1 \Leftrightarrow \nu_\p\left(\binom{1/2}{n}\right) \geq 9.
	\]
Define $f: \Q_p \to \Q_p$ via
	\[
	f(x) = \binom{x}{n}= \dfrac{x(x-1)(x-2)\cdots(x-n+1)}{n!}.
	\]
Note that $f$ is continuous. Now
	\[
	\binom{\frac{p^m+1}{2}}{n}= f\left(\dfrac{p^n+1}{2}\right) \stackrel{m \to \infty}{\longrightarrow} f\left(\frac{0+1}{2}\right)= \binom{1/2}{n}.
	\]
On the left side, we have $\binom{\frac{p^m+1}{2}}{n} \in \Z$. Therefore as $\left| \frac{p^m+1}{2}\right|_\p \leq 1$ for $m \geq 1$, we have $\left|\binom{1/2}{n}\right|_\p \leq 1$. \xqed
\end{ex}


\begin{ex}
Consider the field $\Q_5$. Define
	\[
	\alpha= \dfrac{1}{2} \sum_{n=0}^\infty (-1)^n \binom{1/2}{n} 5^n.
	\]
We first need check that $\alpha$ is well defined: the individual terms tend to 0 as
	\[
	\left| (-1)^n \binom{1/2}{n} \right|_5 \leq 1,
	\]
and as in the previous example $|5^n|_5=5^{-n} \to 0$ as $n \to \infty$. For real $|x|<1$,
	\[
	\sqrt{1+x}= \sum_{n=0}^\infty \binom{1/2}{n} x^n.
	\]
But then we have an equality of formal power series
	\[
	1+x = \left(\sum_{n=0}^\infty \binom{1/2}{n} x^n \right)^2.
	\]
Setting $x= -5$, 
	\[
	-4 = \left(\sum_{n=0}^\infty \binom{1/2}{n} (-5)^n \right)^2 \in \Q_5.
	\]
Therefore, $\alpha^2= -1$. In particular, $\Q_5$ contains a fourth root of unity. 
\end{ex}

Now that our fields possess a notion of metric, we may define the ring of integers in terms of this metric.

\begin{dfn}[Ring of $\p$-adic Integers]
The subring $K_\p$ 
	\[
	\O_\p:= \{x \in K_\p \colon |x|_\p \leq 1\}
	\]
is called the ring of $\p$-adic integers. In the case where $K=\Q$, we denote $\O_\p$ by $\Z_p$.
\end{dfn}

It is routine to verify that $\O_\p$ is indeed a subring of $K_\p$. 

\begin{prop}
$\O_\p$ is a DVR.
\end{prop}

\noindent \emph{Proof (Sketch): }  The function $\nu_\p: K^\times \to \Z \subseteq \R$ is continuous with respect to $|\cdot|_\p$, and extends uniquely to a continuous map $\nu_\p: K_\p^\times \to \Z \subseteq \R$. Choose $\pi \in \p \setminus \p^2$. Note that $\nu_\p(\pi)=1$. Let $\O_\p^\times=\{x \in K_\p \colon |x|_\p=1\}$. For $x \in \O_p^\times$, $|x^{-1}|_\p = |x|_\p^{-1}$ and for $a \in \O_\p \setminus \{0\}$, $\nu_\p(a \pi^{-\nu_\p(a)})=0$. Therefore, $|a\pi^{-\nu_\p(a)}|_\p=1$, and it follows that $a\pi^{-\nu_\p(a)} \in \O_p^\times$. The nonzero ideals of $\O_\p$ are $\pi^n \O_\p$ with $n \geq 0$. Hence, $\O_\p$ is a PID with a unique maximal ideal, i.e. a local PID. Therefore, $\O_\p$ is a DVR. \qed \\


\begin{rem}
We can relate $\O_\p$ to the DVR $\O_K$ via the isomorphism
	\[
	\qfrac{\O_K}{\p} \ma{\sim} \qfrac{\O_\p}{\p \O_\p} = \qfrac{\O_\p}{\pi \O_\p}.
	\]
Note that the left side is a finite field so that $\O_\p/\p\O_\p$ is a finite field also. 
\end{rem}

Parallel to the ordinary number field case, given an element of $K_\p$, we may multiply by a sufficiently large power of $\pi \in \p \setminus \p^2$ to obtain an element of $\O_\p$. Therefore to understand the elements of $K_\p$, one need only understand the elements of $\O_\p$. 

Fix a finite set $S \subseteq \O_K$ representing the cosets of $\O_K/\p$. Fix $\pi \in \p \setminus \p^2$, i.e. $\pi \in \O_K$ and $\nu_\p(\pi)=1$. 


\begin{thm}
Any $x \in \O_\p$ can be written uniquely as $\displaystyle\sum_{n=0}^\infty a_n \pi^n$, where $a_i \in S$. Conversely, any such series converges to an element of $\O_\p$. 
\end{thm}

\noindent \emph{Proof (Sketch): } Let $x \in \O_\p$. We make use of the fact that
	\[
	\qfrac{\O_K}{\p} \ma{\sim} \qfrac{\O_\p}{\p \O_\p} = \qfrac{\O_\p}{\pi \O_\p}.
	\]
Now $x \equiv a_0 \mod \p\O_\p$ for a unique $a_0 \in S$. But then $\frac{x-a_0}{\pi} \in \O_\p$. Then we have $\frac{x-a_0}{\pi} \equiv a_1 \mod \p \O_\p$ for a unique $a_1 \in S$. Then as before, $\frac{x-(a_0+a_1\pi)}{\pi^2} \in \O_\p$. Repeating this process, we find $x-(a_0+a_1\pi+\cdots+a_n\pi^n) \in \pi^{n+1} \O_\p$, where $a_i \in S$ are unique. Finally,
	\[
	\left| x - \sum_{i=0}^n a_i \pi^i \right|_\p \leq |\pi|_\p^{n+1} = \left(\dfrac{1}{N(\p)}\right)^{n+1} \ma{n \to \infty} 0. 
	\] \qed \\


\begin{prop}
The space $\O_\p$ is compact. 
\end{prop}

\noindent \emph{Proof (Sketch): } Since $K_\p$ is a metric space, it suffices to prove that $\O_\p$ is sequentially compact. Consider any sequence $(x_n)_{n \in \N}$ in $\O_\p$. Write each $x_i$ as $\displaystyle x_i = \sum_{n=0}^\infty a_{n_i} \pi^n$ for $a_{n,i} \in S$. There are infinitely many $x_n$ with the same $a_{0,i}$ as $S$ is finite. For such $x_n$, there are infinitely many with the same $a_{1,i} \in S$. Continuing in this fashion, we obtain a convergence subsequence in $\O_\p$. \qed \\

Note that for any $a \in K$, $a+ \O_\p$ is an open neighborhood of $a$ that is compact. Hence, $K_\p$ is a local field. 





\subsection{Extensions of $\Q_p$}

Fix a prime $p$, and let $K/\Q_p$ be a finite field extension. Let $B$ be the integral closure of $\Z_p$ in $K$.
	\[
	\begin{tikzcd}
	K \arrow[draw=none]{r}[sloped,auto=false]{\supseteq} \arrow[dash]{d}  & B \arrow[dash]{d} \\
	\Q_p \arrow[draw=none]{r}[sloped,auto=false]{\supseteq} & \Z_p
	\end{tikzcd}
	\]
We know that integral closure of Dedekind domains are Dedekind so that $B$ is Dedekind. But then $pB$ factors uniquely, say $pB= \P_1^{e_1} \cdots \P_r^{e_r}$. In fact, there is a single prime so that $pB=\P^e$. Then $[K \colon \Q_p]= ef$, where $f=[B/\p \colon \Z_p/(p)]$. Furthermore, we have valuations
	\[
	\begin{split}
	\nu_\p&: K^\times \to \Z \\
	\nu_\p&: \Q_p^\times \to \Z
	\end{split}
	\]
such that $\nu_\p\big|_{\Q_p^\times}= e \nu_\p$. 

\begin{prop}\label{prop:extend}
The $p$-adic absolute value $|\cdot|_p$ extends uniquely to $K$.
\end{prop}

\begin{cor}
The $p$-adic absolute value extends uniquely to $\overline{\Q}_p=\overline{K}$. 
\end{cor}


%\begin{lem}[Krasner's Lemma] \label{lem:kras}
%Let $\alpha,\beta \in \overline{K}$ and let $p_\alpha(x) \in K[x]$ be the minimal polynomial of $\alpha$.
%
%
%\end{lem}

\pf Choose $\sigma: K(\alpha,\beta) \hra \overline{K}$ that fixes $K(\beta)$. It suffices to show that $\sigma(\alpha)=\alpha$. As $\beta=\sigma(\beta)$, we have
	\[
	|\sigma(\alpha) - \beta|_p = |\sigma(\alpha) - \sigma(\beta)|_p = |\sigma(\alpha-\beta)|_p.
	\]
Now $|\sigma(\cdot)|_p$ is an absolute value on $K(\alpha,\beta)$ which extends the absolute value $|\cdot|_p$ on $\Q_p$. By Proposition~\ref{prop:extend}, this extension is unique. Therefore, $|\sigma(\cdot)|_p=|\cdot|_p$. Observe that
	\[
	|\sigma(\alpha) - \beta|_p = |\sigma(\alpha)-\sigma(\beta)|_p = |\sigma(\alpha-\beta)|_p= |\alpha-\beta|_p.
	\]
Therefore,
	\[
	\begin{split}
	|\sigma(\alpha)-\alpha|_p&= |\sigma(\alpha) - \beta+\beta-\alpha|_p \\
	&\leq \max\{|\sigma(\alpha)-\beta|_p, |\beta-\alpha|_p\} \\
	&=|\sigma(\alpha) - \beta|_p.
	\end{split}
	\]
As $|\sigma(\alpha)-\beta|< |\sigma(\alpha)-\alpha|$ for $\sigma(\alpha) \neq \alpha$, we must have $\sigma(\alpha)=\alpha$. \qed \\

Krasner's Lemma is an amazing result that links the analytic properties of local fields with algebraic properties of the field. For instance, consider the following result:

\begin{prop}\label{prop:close}
Fix a monic, irreducible polynomial $f(x) \in K[x]$ of degree $n$. For any $g(x) \in K[x]$ of degree $n$ ``sufficiently close'' (with respect to $|\cdot|_p$) to $f(x)$, $g(x)$ is irreducible and for any root $\alpha \in \overline{K}$ of $f$, there is a root $\beta \in \overline{K}$ of $g$ such that $K(\alpha)=K(\beta)$. 
\end{prop}

From this, we are able to obtain the following:

\begin{prop}
There is a number field $L$ and a prime $\p \subseteq \O_L$ dividing $p$ such that $K=L_\p$. 
\end{prop}

\noindent \emph{Proof (Sketch): } Using the Primitive Element Theorem, $K=\Q_p(\alpha)$. Let $f(x) \in \Q_p[x]$ be the minimal polynomial of $\alpha$ over $\Q_p$. Take $g(x) \in \Q[x]$ ``sufficiently close'' to $f(x)$. By Proposition~\ref{prop:close} since $\Q$ is dense in $\Q_p$, there is a root $\beta \in \overline{\Q} \subseteq \overline{\Q}_p$ of $g(x)$ such that $\Q_p(\alpha)=\Q_p(\beta)$. Take $L=\Q(\beta)$. \qed \\


\begin{thm}
The local fields $K$ of characteristic zero are (up to isomorphism)
\begin{enumerate}[(i)]
\item finite extensions $K/\Q_p$ 
\item $\R$ or $\C$
\end{enumerate}
The local fields $K$ of positive characteristic are (up to isomorphism) $\F_q(\!(x)\!)$ for $q$ a prime power.
\end{thm}

If $B$ is the integral closure of $\Z_p$ in $K/\Q_p$, then $B=\O_\p$ for a prime ideal $\p \subseteq \O_L$, where $K=L_\p$. 


\begin{lem}[Hensel's Lemma] \label{lem:hen}
Let $f(x) \in B[x]=\O_p[x]$ be a monic polynomial and $\overline{f}(x) \in \F_p[x]$ be its reduction mod $\p$. If $a \in \F_p$ is a simple root of $\overline{f}$, then there is a unique $\alpha \in \O_\p$ with $\alpha \equiv a \mod \p$ such that $\alpha$ is a root of $f(x)$. 
\end{lem} 

\noindent \emph{Proof (Sketch): } Suppose $\alpha_n \in \O_\p$ is such that $\alpha_n \equiv a \mod \p$ and $f(\alpha_n) \equiv 0 \mod \p^n$ (note that this is true if $n=0$). Take $\pi \in \O_\p$ with $\nu_\p(\pi)=1$. We want to solve 
	\[
	0 \equiv f(\alpha_n+b\pi^n) \equiv f(\alpha_n) + f'(\alpha_n)b\pi^n \mod \p^{n+1}
	\]
Equivalently, we want to solve
	\[
	f'(\alpha_n)b \equiv - \dfrac{f(\alpha_n)}{\pi^n} \mod \p.
	\]
We know that $f'(\alpha_n)b \equiv f'(a)b \mod \p$ and $f'(a)b \not\equiv \mod \p$ as $a$ is a simple root. But then we may solve the previous equation for $b \in \O_\p$. Then $\alpha_{n+1}=\alpha_n+b\pi^n$. The sequence $(\alpha_n)_{n \in \N}$ converges to a root. \qed \\

Note that the proof of Hensel's Lemma even gives an algorithm for finding a root! Now let $K/\Q_p$ be a finite extension of fields with ring of $\p$-adic integers $\O_\p \subseteq K$. We have a field extension
	\[
	\begin{tikzcd}
	\O_\p/\p \arrow[dash]{d}{f} \\
	\Z_p/(p) \cong \F_p 
	\end{tikzcd}
	\]
where $f=[\O_\p \colon \Z_p/(p)]$. But then $\O_\p/\p$ is the splitting field of $x^{p^f}-x \in \F_p[x]$. Since this polynomial is separable over $\F_p$, Hensel's Lemma gives that $x^{p^f}-x$ is separable in $\O_\p[x]$. But then $K$ contains $p^f-1$ roots of unity. Note that this was `difficult' to show in the Number Field case whereas in the local field case we essentially get roots of unity ``for free'' by Hensel's Lemma. 


Let $\mu_{p^f-1} \subseteq K$ denote the roots of unity in $K$. We have a tower of field extensions
	\[
	\begin{tikzcd}
	K \arrow[dash]{d}{e} \\
	\Q_p(\mu_{p^f-1}) \arrow[dash]{d}{f} \\
	\Q_p
	\end{tikzcd}
	\]
The extension $K/\Q_p(\mu_{p^f-1})$ is totally ramified, but the extension $\Q_p(\mu_{p^f-1})/\Q_p$ is unramified. 

\begin{prop}\label{prop:finqpext}
Fix a prime $p$ and integer $n \geq 1$. There are only finitely many extensions, up to isomorphism, $K/\Q_p$ of degree $n$.
\end{prop}

\noindent \emph{Proof (Sketch): } Any extension $K/\Q_p$ is a totally ramified extension of the intermediate field $\Q_p(\mu_{p^f-1})$ of degree $e$. Note that both $f$ and $e$ divide $n$. Set $F:= \Q_p(\mu_{p^f-1})$. We need only show that there are only finitely many totally ramified extensions $K/F$ of degree $e$.

Choose a uniformizer $\pi \in K$ with $\nu_K(\pi)=1$. The minimal polynomial of $\pi$ over $F$ is Eisenstein at $\p \subseteq \O_\p$. By Proposition~\ref{prop:close}, `small' change in the coefficients of this minimal polynomial does not change the field extension. This gives an open cover of the compact set $\p^{e-1}(\p - \p^2)$. There is then a finite subcovering so that $K$ can be obtained from one of these finitely many Eisenstein polynomials. \qed \\

This even gives information about number fields.

\begin{thm}
Let $K$ be a number field, $S$ a finite set of primes of $\O_K$, and $n \geq 1$ an integer. There are only finitely many extensions $L/K$ of degree $n$ unramified at all primes $\p \notin \O_K$. 
\end{thm}

\noindent \emph{Proof (Sketch): } Consider the case $K=\Q$. We know there are only finitely many extensions $L/\Q$ of degree $n$ with given discriminant $\disc L$. The prime divisors of $\disc L$ are the ramified primes of the extension $L/\Q$. We may bound the powers of $\disc L$ occurring by using the finiteness of the number of extensions of $\Q_p$ from Proposition~\ref{prop:finqpext}. \qed \\





\subsection{Class Field Theory} 

































